\section{Introduction}
The Jaynes Cummings model has been the subject of many different theoretical investigations in many different guises
\cite{Abdalla2011}
\cite{Benivegna1994}. 
Coupled qubit cavity systems of this sort are interesting in application, and have been investigated in the context of open quantum systems, quantum measurement, and quantum information and computing. One particular, promising implementation is that of circuit QED 
\cite{Blais2004a}
where there are promising proposals for reaching the strong coupling regime of QED, for concrete investigations of the theory of open quantum systems, quantum measurement, and quantum information processing
\cite{You2003}
\cite{Hood2000}
\cite{Irish2003}
. This context also provides a base for broader investigation of effects in cavity QED 
\cite{Al-Saidi2002}
\cite{Plastina2003}
\cite{Marquardt2001}
. Here we investigate the driven Jaynes Cummings Model using the typical parameter ranges of circuit QED, with a particular focus on bistabilities in the dispersive regime i.e. where the qubit is far from the cavity resonance. In the first section we review theoretical background. Next, we review recent developments, first in the resonant regime
\cite{Carmichael2015}, 
before moving to the dispersive regime 
\cite{Bishop2010}. 
Finally, we present our results.

