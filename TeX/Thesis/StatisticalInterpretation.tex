\part{Background}
\section{The Statistical Interpretation of Quantum Mechanics\cite{Breuer2002}}
Consider a statistical ensemble $\varepsilon$ consisting of a suitably large number N of identically prepared quantum systems
\begin{equation}
  \varepsilon = \{S^1, S^2, \ldots, S^N\}
\end{equation}
Such a statistical ensemble represents a particular set of experimental conditions, whose realisation in each instance generates a particular member $S^i$ of  $\varepsilon$

The first postulate of the statistical interpretation of quantum mechanics is that a complete characterisation of such a statistical ensemble can be represented by a normalised state vector $| \psi \rangle$ in an associated Hilbert space $\mathcal{H}$

The second postulate is that all the possible outcomes of measurements on such an ensemble are represented by self-adjoint operators on $\mathcal{H}$. The outcome of the measurement corresponding to an operator $\hat{R}$ represents a real valued random variable R with cumulative distribution function $F_{\hat{R}}$ defined via the family of orthogonal projection operators that constitute the spectral decomposition of $\hat{R}$:
\begin{equation}
  \hat{R} = \int_{-\infty}^\infty rdE_r
\end{equation}
as
\begin{equation}
  F_{\hat{R}} = \langle \psi |E_r | \psi \rangle
\end{equation}
This characterisation of the possible statistical variation of a quantum system is not yet complete, for it neglects classical uncertainty associated with which statistical ensemble $\varepsilon$ represents a given quantum system. To generate the most general possible representation of a quantum statistical ensemble, a number of possible quantum ensembles $\varepsilon_\alpha$ of the above type are mixed with weights $w_\alpha$ (these weights are probabilities in the classical, not quantum sense). A self-adjoint operator now yields a random variable R with cumulative distribution function
\begin{equation}
  F_{\hat{R}} = \sum_\alpha w_\alpha \langle \psi_\alpha | E_r | \psi_\alpha \rangle
\end{equation}
for convenience, the \emph{density operator} $\rho$ can be introduced thus
\begin{equation}
  \dens = \sum_\alpha w_\alpha |\psi_\alpha \rangle \langle \psi_\alpha |
\end{equation}
and the cumulative distribution written:
\begin{equation}
  F_{\hat{R}} = tr\{E_r \dens \}
\end{equation}
by a similar mixing argument and using the spectral decomposition of $\hat{R}$ the mean and variance of the random variable associated with $\hat{R}$ as above are as follows
\begin{align}
  \langle \hat{R} \rangle &= tr\{\hat{R} \dens \} \\
  var(\hat{R}) &= \langle \hat{R}^2 \rangle - {\langle \hat{R} \rangle}^2
\end{align}
The density matrix constitutes a complete description of the statistical properties of an open quantum system. Thus, the dynamics of such a system can be described by the time evolution of its density operator.
