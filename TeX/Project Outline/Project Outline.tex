%Title Data
\documentclass[reqno]{amsart}
\title{Project Outline}
\author{Fergus Barratt}

%Packages
% \usepackage{amsthm}
% \usepackage{mathrsfs}
\usepackage[margin=1in]{geometry}
% \usepackage{fullpage}
% \usepackage{graphicx}
% \usepackage[backend=biber, style=ieee]{biblatex}
% \usepackage{todonotes}
% \usepackage{graphicx}
% \usepackage{float}
% \usepackage{caption}
% \usepackage{subcaption}
% \usepackage{grffile}
% \graphicspath{ {./Images/} }

\begin{document}
\maketitle

This project investigates a phase bistability in the driven Jaynes Cummings model, and its representation in terms of quasiprobability functions. 

By performing computational simulations and analytical models we plan to investigate the Jaynes-Cummings model from this viewpoint, with particular focus on the dispersive case - i.e. when the cavity and qubit are far detuned from one another. 

\section{Timeline}
Until Christmas 2015 we will be familiarising ourselves with the driven Jaynes-Cummings model in the resonant case, starting with the paper by Carmichael that is summarised in the Literature Review. Beyond that, we plan to move to the dispersive regime, and compare results from this methodology with those produced by the other Masters student on the project who is producing Monte Carlo simulations of the same system.

% By the end of this project familiarity with the Markovian quantum master equation will be acquired along with the quasi-probability formulation of quantum optics through the positive P-representation approach. The role of the cavity as a meter will be investigated by performing Von Neumann projective measurements on the qubit observable in the strong coupling regime, through the correlation of photon pointer states and the qubit eigenstates.

\end{document}