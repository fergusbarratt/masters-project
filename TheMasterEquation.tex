\section{The Quantum Master Equation}
The quantum master equation is in a sense a generalisation of the classical master equation, which describes the time evolution of a system confined to a set of states via a set of differential equations for the probability of the system occupying each state. In the quantum case, differential equations not just for the probability of occupying a particular state (the diagonal elements of $\dens$) are required, but also equations describing how the off diagonal elements of the density matrix evolve in time because of their relation to quantum coherence effects.

In the nonrelativistic theory of quantum mechanics the time evolution of the pure state state vector is described by the \emph{Schr\"odinger equation}:
\begin{equation}
	i\hbar \frac{d}{dt} |\psi(t)\rangle = \ham(t) |\psi(t) \rangle
\end{equation}

with $\ham$ the Hamiltonian of the system. Setting $\hbar$ to 1 going forwards, the time dependence in the above can also be represented in terms of a time evolution operator (generated by the Hamiltonian) 

\begin{equation}
	|\psi(t) \rangle = U(t, 0) | \psi(0) \rangle
\end{equation}

by substitution of the (10) into (9), it is easy to see that $U^{\dagger}U = UU^\dagger = I$ provided $\ham$ is hamiltonian i.e. U represents unitary time evolution of the system

For a mixed state of a closed system, an equation of motion for the density matrix
\begin{equation}
	\dens (t) = \sum_\alpha w_{\alpha} |\psi_\alpha (t) \rangle \langle \psi_\alpha (t) |
\end{equation}
can be found by propagating each normalised $| \psi_\alpha (t) \rangle$ with the unitary time evolution operator given by solving the Schr\"odinger equation, the net result of which is more concisely expressed
\begin{equation}
	\dens (t) = U(t, 0)\dens (0) U^\dagger (t, 0)
\end{equation}
which, when differentiated with respect to time yields the \emph{Liouville-Von Neumann} equation
\begin{equation}
	\frac{d}{dt}\dens(t) = i [\ham, \dens (t) ]
\end{equation}

Non-unitary dynamics of open systems result from partitioning a larger, unitarily evolved system into system and environment, and performing a partial trace over the environment degrees of freedom.

We start with the Hamiltonian for a closed, mixed quantum system, potentially time dependent, and partition it into components representing: a subsystem constituting the totality of the interesting dynamics: the \emph{system}, a subsystem representing the dynamics of the remaining subsystem the \emph{environment} or \emph{reservoir}, and a component representing the interaction between system and environment (there is an associated partitioning of the Hilbert space into the tensor product of system and environment Hilbert spaces, and in the equation below it is understood that operators with the subscript S operate only on the system degrees of freedom i.e. exist in the system Hilbert space and represent identity in the environment Hilbert space, and etc.)
\begin{equation}
	\ham = \ham_S +\ham_R + \ham_I
\end{equation}

The partial trace operation is an operator-valued function that takes a operator on a larger Hilbert space and discards its action on all but a smaller Hilbert space, colloquially "tracing over" the degrees of freedom of the discarded subsystem to leave the operator on a smaller subsystem.

Starting from equation (12) and tracing over the environment
\begin{equation}
	\dens_S (t) = tr_E \{U(t, 0) \dens_S (0) U^\dagger (t, 0) \}
\end{equation}
with $\dens_S = tr_E\{\dens\}$ we arrive at the most general form for the Liouville-Von Neumann equation for the \emph{reduced} system, also known as the reduced master equation
\begin{equation}
	\frac{d}{dt} \rho_S (t) = -itr_E\{[\ham(t), \dens(t)]\}
\end{equation}
A marked simplification is derived by invoking an approximation known as the Markov Approximation. 

The most general form\autocite[119--122]{Breuer2002} for the reduced master equation in the Markov approximation is known as the \emph{Lindblad Form}
\begin{equation}
	\frac{d}{dt} \rho_S (t) = -i[\ham (t), \dens (t)] + \sum_{k=1}^{N^2-1} \gamma_k \{ \hat{A}_k \dens_S \cre_k -\frac{1}{2}  \cre_k \dens_S \hat{A}_k -\frac{1}{2} \dens_S \cre_k \hat{A}_k \}
\end{equation} 
with $\{A_k\}_{k \in I}$ some indexed complete orthonormal basis of operators on the system Hilbert space

