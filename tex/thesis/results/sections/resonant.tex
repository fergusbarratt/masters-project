%% resonant results
\section{Results: Resonant}
We investigate the resonant regime, following ref \cite{Carmichael2015}, by computational solution of the master equation, mean-field methods, and monte-carlo simulations.
\subsection{Quantum: Computational}
We have the hamiltonian 
\begin{equation}
  \ham = \delta_{cd} \cre \ann + \delta_{qd} \sigma_z + g \left(\cre \atann + \atcre \ann \right) + \xi \left(\ann + \cre \right)
\end{equation}
in the master equation
\begin{equation}
  \dot{\dens} = -i \comm{\ham}{\dens} + \lindblad{\kappa}{\dens} + \lindblad{\gamma}{\dens}
\end{equation}
We truncate the density matrix expansion at half the expected intracavity photon number, set the left hand side to zero and solve the resulting matrix equation in software (see the appendix).

We also computationally calculate Q and W functions.
\subsection{Quantum: Analytical}
With the system on resonance and in the strong coupling regime, the system does not permit any kind of system size or perturbation expansion \cite{Carmichael2015}.

\cite{Alsing1992} derive a canonical transformation of the classically driven Hamiltonian which diagonalises the system on qubit-cavity resonance with (quasi-)energies.
\begin{align}
  e_{n, +} &= + \sqrt{n} \hbar g {\left \{1 - {\left ({\frac{2\xi}{g}} \right )}^2 \right \}}^{\frac{3}{4}} \\
  e_{n, -} &= - \sqrt{n} \hbar g {\left \{1 - {\left ({\frac{2\xi}{g}} \right )}^2 \right \}}^{\frac{3}{4}}
\end{align}
We see a critical point at  $\frac{2\xi}{g} = 1$, where the quasienergy splitting collapses to zero.
\subsection{Semiclassical}
We investigate the semiclassical equations for the system.
\begin{align}
  &\frac{d \alpha}{dt} = -(\kappa -i \Delta \omega) \alpha-ig \beta \label{eq:alpha}\\
  &\frac{d \beta}{dt} = i \Delta \omega \beta +ig \alpha \zeta \label{eq:beta}\\
  &\frac{d \zeta}{dt} = 2 i g(\alpha^* \beta -\alpha \beta^*)\label{eq:zeta}
\end{align}
and
\begin{equation}
  4|\beta|^2+\zeta^2 = 1 \label{eq:pseudospin}
\end{equation}
in the absence of spontaneous emission $\gamma=0$
\footnote{ Interestingly, setting $\gamma$ equal to zero at this point yields different asymptotic solutions to those for the system with $\gamma$ included and set to zero in the solution \cite{Alsing1990}},
A derivation can be found in the appendix.
\begin{figure}[htb]
  \includegraphics[width=\linewidth]{sc_no_det.pdf}
  \caption{$\alpha$ and $\zeta$ as the drive strength $\xi$ moves up to through and beyond the critical point (a) $\zeta$ approaches zero with increasing drive strength (positive branch)\label{fig:zeta} (b) development of field phase bistability}\label{fig:alpha}
  \label{fig:sc_no_det}
\end{figure}
\subsubsection{Neoclassical Radiation Theory}
In the absence of drive and detuning and with the cavity field derivative set to zero
\begin{align}
  & 0 = -\kappa \alpha - ig \beta \\
  \implies & \alpha = \frac{ig}{\kappa} \beta
\end{align}
in \cref{eq:zeta}
\begin{equation}
  \frac{d \zeta}{dt} = -4 g^2 |\beta|^2
\end{equation}
and from \cref{eq:pseudospin}
\begin{align}
   |\beta|^2 &= (1-\zeta^2)/4 \\
\implies \frac{d \zeta}{dt} &= -\frac{g^2}{\kappa} (1-\zeta^2)
\end{align}
We recover the non exponential decay of neoclassical radiation theory
\subsubsection{Steady State}
We now set all derivatives to zero, and consider the asymptotic solutions to the mean-field equations, which must satisfy:
\begin{align}
  -ig \beta -i \xi &= 0 \\
  ig\alpha \zeta &= 0
\end{align}
from which are obvious two branches of solutions $\rightarrow \zeta = 0$ or $\alpha = 0$. We take $\alpha = 0$ and from \cref{eq:alpha} and \cref{eq:pseudospin}
\begin{align}
  \beta &= -\frac{\xi}{g} \\
  \zeta &= \mp \sqrt{1 - {\left( \frac{2\xi}{g} \right)}^2}
\end{align}
Increasing drive through the critical point $\xi = \frac{g}{2}$ the difference under the square root becomes negative and the inversion $\zeta$ imaginary and unphysical. We take up the other branch $\zeta = 0$, and from \cref{eq:alpha}
\begin{align}
  \beta &= \pm \frac{\alpha}{2|\alpha|} \\
  \zeta &= 0
\end{align}
with $\alpha$ a solution to
\begin{equation}
  \alpha = -i \xi{\left ( \kappa \pm i \frac{g}{2|\alpha|} \right )}^{-1}
  \label{eq:alphacondnotdet}
\end{equation}
in \cref{fig:sc_no_det} the phase bistability above the critical point is obvious. The two $\beta$ solution branches start coincident in phase ($\pi$ and $-\pi$) at drive strengths just above critical and very quickly move to opposite sides of the Bloch sphere ($-\frac{\pi}{2}$ and $\frac{\pi}{2}$).

The phase of $\beta$ above the critical point follows the phase of $\alpha$, either aligned or antialigned. This spontaneous development of phase bistability Alsing and Carmichael call `Spontaneous Dressed State Polarisation' \cite{Alsing1990}. Referred to the fully quantum model, each of the two phases corresponds to a the system ascending different sets of rungs of the Jaynes Cummings ladder, either $\ket{E_{n, U}}$ or $\ket{E_{n, L}}$
\subsubsection{Non-zero detuning}
Solving \cref{eq:alpha}, \cref{eq:beta}, \cref{eq:zeta} in the steady state gives the following self consistency equation for $\alpha$
\begin{equation}
  \alpha = i \xi\frac{1}{\left[\kappa-i\left(\Delta \omega \mp \operatorname{sgn}(\Delta \omega) \frac{g^2}{\sqrt{\Delta \omega^2 +4g^2 |\alpha|^2}}\right)\right]}
\end{equation}
and for $\beta$ and $\zeta$
\begin{align}
  \beta& = \pm \operatorname{sgn}(\Delta \omega) \frac{g \alpha}{\sqrt{\Delta \omega^2 + 4 g^2 |\alpha|^2}}\\
  \zeta& = \mp \sqrt{1-4|\beta|^2}
\end{align}
for $\beta$ and $\zeta$, where $\operatorname{sgn}(\Delta \omega)$ is the sign of the detuning with $\operatorname{sgn}(0) = 1$.
The expression for $\alpha$ is a lorentzian, with dispersive sign dependent shift $ \operatorname{sgn}(\Delta\omega)\flatfrac{g^2}{\sqrt{\Delta\omega^2+4g^2\abs{\alpha}^2}}$, which diverges on resonance as alpha goes to zero. 
In the limit $\alpha \rightarrow \infty$ the system response returns to the cavity lorentzian.

\subsection{Spontaneous Emission}
\begin{figure*}[bht]
  \includegraphics[width=\linewidth]{amplitudebistability.pdf}
  \caption{Development of amplitude bistability in Q function. Parameters $\omega_c=10,\ \omega_q=10,\ \xi=4,\ \kappa=1,\ g=10$. Value of $\omega_d$ marked.}
  \label{amplitudebistability}
\end{figure*}

We now reintroduce the spontaneous emission parameter.
The optical Bloch equations with spontaneous emission read (see \cref{sc_with_gamma})
\begin{align}
  \frac{d \alpha}{dt} &= -(\kappa - i \Delta \omega)\alpha - ig \beta - i\xi\label{eq:alphase}\\
  \frac{d \beta}{dt} &= -(\frac{\gamma}{2}-i\Delta\omega)\beta+ig\alpha\zeta \label{eq:betase}\\
  \frac{d\zeta}{dt} &= -\gamma (\zeta +1)+2ig(\alpha^*\beta-\alpha\beta^*) \label{eq:zetase}
\end{align}
In the steady state \cref{eq:alphase} becomes
\begin{align}
  \beta &= \frac{ig\alpha\zeta}{\frac{\gamma}{2}-i\Delta\omega}
\end{align}
from which in \cref{eq:zetase} in steady-state
\begin{align}
  0 &= -\gamma(\zeta+1)+2ig\frac{ig|\alpha|^2\zeta\frac{\gamma}{2}2}{\flatfrac{\gamma^2}{4}+\Delta\omega^2} \\
  \implies \zeta &= \frac{1}{\frac{-2g^2|\alpha|^2}{\flatfrac{\gamma^2}{4} +\Delta\omega^2}-1} \\
  &= \frac{1}{\frac{-2g^2|\alpha|^2 - \flatfrac{\gamma^2}{4}-\Delta\omega^2}{\flatfrac{\gamma^2}{4} +\Delta\omega^2}}
\end{align}
putting the above together yields
\begin{align}
  \beta &= \frac{ig\alpha\zeta}{\frac{\gamma}{2}-i\Delta\omega}\\
  &= \frac{ig\alpha}{\frac{(-2g^2|\alpha|^2-\flatfrac{\gamma^2}{4}-\Delta\omega^2)(\frac{\gamma}{2}-i\Delta\omega)}{\flatfrac{\gamma^2}{4}+\Delta\omega^2}}\\
  &= \frac{ig\alpha}{\frac{-2g^2|\alpha|^2-\flatfrac{\gamma^2}{4}-\Delta\omega^2}{\frac{\gamma}{2}+i\Delta\omega^2}}\\
  &= \frac{ig\alpha(\frac{\gamma}{2}+i\Delta\omega)}{-2g^2|\alpha|^2-\flatfrac{\gamma^2}{4}-\Delta\omega^2} \label{eq:betasolved}
\end{align}
\cref{eq:alphase} in the steady state with  \cref{eq:betasolved} becomes a condition for $\alpha$
\begin{align} % Here be weird bugs
  0&=-(\kappa-i\Delta\omega)\alpha-ig\beta-i\xi \\
  0&=-{(\kappa-i\Delta\omega)}\alpha-g\frac{ig(\frac{\gamma}{2}+i\Delta\omega)}{-2g^2|\alpha|^2-\flatfrac{\gamma^2}{4}-\Delta\omega^2}\alpha-i\xi \\
\implies \alpha &= -i\xi \frac{1}{\kappa-i\Delta\omega+\frac{g^2(\frac{\gamma}{2}+i\Delta\omega)}{\flatfrac{\gamma^2}{4}+{\Delta\omega}^2+2g{|\alpha|}^2}}\label{eq:alphadetdiss}
\end{align}

\Cref{eq:alphadetdiss} is the classical solution for the steady state of a saturable two level transition, with a saturation photon number ($I \propto |\alpha|^2 \approx n_{sat}$) of $n_{sat} = \flatfrac{\gamma^2}{8g^2}$.  \\
Compare $\alpha$ derived with $\gamma$ present to without
\begin{widetext}
  \begin{align}
    \alpha(\kappa, \gamma) &= \flatfrac{-i\xi}{\left[ \kappa + \frac{g^2\gamma/2}{\gamma^2/4+\Delta\omega+2g\abs{\alpha}^2} - i\left( \Delta\omega+\frac{g^2\Delta\omega}{\gamma^2/4+\Delta\omega^2 + 2g\abs{\alpha}^2}\right)\right]}\\
            \alpha(\kappa) &= \flatfrac{-i\xi}{\left[\kappa-i\left(\Delta \omega - \frac{g^2}{\sqrt{\Delta \omega^2 +4g^2 |\alpha|^2}}\right)\right]}
  \end{align}
\end{widetext}
where the functional dependence on the collapse parameters has been shown.
Here, the lorentzians are broadened as well as shifted, by a factor $\flatfrac{g^2\gamma/2}{(\gamma^2/4+\Delta\omega + 2g\abs{\alpha}^2)}$
\subsubsection{Difference between limits}
Setting $\gamma$ and $\Delta\omega$ to zero, the condition becomes
\begin{equation}
  \alpha = -i\xi\frac{1}{\kappa}
\end{equation}
which is notably not the same as \cref{eq:alphacondnotdet}.\\

The presence of $\gamma$ in the Maxwell-Bloch Equations changes the asymptotic solutions, even if $\gamma$ is set to zero in these solutions.
The conservation law $4|\beta|^2 +\zeta^2 = 1$ is broken by the qubit relaxation parameter $\gamma$.
The solutions in the case that the limit is taken after steady state requirement is imposed are those of absorptive optical bistability.
In the limit $\frac{\gamma}{\kappa} \rightarrow 0$ the rate at which these steady states are approached becomes vanishingly small
 \cite{Alsing1990}.
 \subsection{Analysis: Resonant Phase Transitions}
We have two regions of bistability.
\subsubsection{Amplitude}
\begin{figure*}[tbh]
  \includegraphics[width=\linewidth]{phasebistability.pdf}
  \includegraphics[width=\linewidth]{phasebistability_with_gamma.pdf}
  \caption{Development of phase bistability in Q function. Parameters $\omega_c = \omega_q = \omega_d ,\ \kappa=1,\ g=10$. Value of $\xi$ marked. (a) With spontaneous emission (b) without spontaneous emission}
  \label{phasebistability}
\end{figure*}
\begin{figure*}[bt]
  \includegraphics[width=\linewidth]{critical_slowing.pdf}
  \caption{Time dependent solutions to the master equation for different parameters. Note the slow asymptotic approach of the green line to the steady state}
  \label{critical_slowing}
\end{figure*}
Here in order to break down photon blockade, we self consistently detune the driving field to address the split resonance. 
The cavity field amplitude is a split lorentzian in the detuning, broadening wihth increasing drive, with a central valley along which the intracavity field increases smoothly as the drive is scanned through the critical point.
Along the inner walls of this lorentzian (the outer walls of the valley) is a region of amplitude bistability. 
There are two coexistent field states as we detune the drive towards the new resonance, the vacuum state where the photon blockade has remained in place, and a displaced squeezed state at nonzero amplitude, where the sufficient detuning has allowed the cavity to fill with light. 
The transition from the dark to the bright cavity represents a single site phase transition, of first order \cite{Carmichael2015}
Here, photon blockade breaks down discontinuously, the response of the cavity changes sharply from the unfilled to the filled state. 
In \cref{amplitudebistability} we plot the Q function of the cavity field for a range of detunings, below the critical drive, without the presence of spontaneour emission.
We see the second stable steady state develop with continouusly increasing probability amplitude. 
Any of the other quasi-probability representations show the same things, the Q and Wigner being the most computationally accessible. 
\subsubsection{Phase}
It is possible to break down photon blockade without detuning the drive.

We have demonstrated \cite{Carmichael2015} the existence on resonance of a critical point in the drive at $2\flatfrac{\xi}{g}=1$, where the quasienergy splitting of the transformed hamiltonian of \cite{Alsing1992} collapses to zero, and the semiclassical inversion becomes imaginary. 

As the drive is scanned through this critical point, in the semiclassical case there develop two solutions from one.
In \cref{phasebistability} we see that the Q function has two peaks, both with the same intracavity amplitude but with different field phase. 
As the drive is scanned through critical, the system spontaneously takes one of the two probable states because of quantum fluctuations. 

We saw how the cavity harmonic ladder was split into two distinct ladders with drive dependent splitting. 
\cite{Carmichael2015} demonstrates that this bistability in phase corresponds to the ascension of distinct ladders by photons. 
Past the critical point, the resolution of the system split energy levels prevents switching between ladders, and there are thus the two stable split states persist aymptotically. 
Spontaneous emission serves to couple the distinct ladders across the energy splitting. 
Here we have switching between the probability peaks, and this switching continues for all time. 
We have then in the steady state Q functions \cref{phasebistability} a connecting fringe between the two phase peaks, where probability transfers between the peaks. 
\begin{figure*}[htb]
    \includegraphics[width=\linewidth]{resonant.pdf}
    \caption{(a) Intracavity photon number in the semiclassical approximation (upper, stable solution) (b) Intracavity photon number, fully quantum, with a field Hilbert space truncated at 85 excitations (c) Q functions with increasing drive on resonance}
    \label{resonant}
\end{figure*}
