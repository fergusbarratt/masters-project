%% dispersive results
\section{Strong Dispersion}
\begin{figure*}[ht]
    \label{dispersive}
    \includegraphics[width=\linewidth]{dispersive.pdf}
    \caption{(a) Squared intracavity cavity amplitude in the semiclassical approximation. Top and bottom lines represent stable states, centre metastable. (b) Absolute intracavity amplitude, in the quantum regime, with a field Hilbert space truncated at 85 excitations. (c) Difference between factorised and unfactorised correlation functions}
\end{figure*}
We now move to the dispersive regime, where the cavity qubit detuning $\delta_{cq}$ is large compared to the other frequencies in the problem. 
We also make the bad cavity assumption, that is $\kappa \gg \gamma, \gamma_\varphi$, where $\gamma_\varphi$ is the dephasing rate, which we assumed to be suppressed so much as to be negligible.
The full hierarchy of scales goes
\begin{equation}
  \gamma, \gamma_\varphi \ll \kappa \ll \frac{g^2}{\delta} \ll g \ll \delta \ll \omega_c
\end{equation}
In this regime, there are several experimentally accessible schemes for qubit quantum non-demolition readout and control \cite{Blais2004a}, without restricting coherent control of the qubit.
One such scheme is described in \cref{disp_QND_readout} 

\subsection{Quantum}
We perform the canonical transformation of \cite{Carbonaro1979}, dropping small terms according to the hierarchy of scales.
\begin{align*}
  \mathscr{N} &= \cre \ann + \sigma_z/2+ 1\\
  \mathscr{T} &= \exp{-\frac{\theta}{\sqrt{4\mathscr{N}}} (\ann \atcre + \cre \atann)}\\
  \text{with }&\\
  \sin(\theta) &= \frac{-2g\sqrt{\mathscr{N}}}{\delta_{cq}}\\
  \cos(\theta) &= \frac{\delta_{cq}}{\Delta}\\
  \Delta &= \sqrt{\delta_{cq}^2+4g^2 \mathscr{N}}
\end{align*}
performing the transformation, and neglecting terms except those $O\left(\mathscr{N}\right)$
\begin{align*}
  &\mathscr{H} = \omega_c \cre \ann + \omega_q \sigma_z + g(\ann\atcre + \cre \atann) + \frac{\xi}{\sqrt{2}} (\ann + \cre ) \cos(\omega_d t) \\
  &\mathscr{T}\mathscr{H}\mathscr{T}^\dagger \approx \omega_c \cre \ann + ( \omega_c - \Delta ) \sigma_z /2 + \frac{\xi}{\sqrt{2}} (\ann + \cre ) \cos(\omega_d t)
\end{align*}
which is solvable but for $\Delta$ in which the excitation number operator $\mathscr{N} = a ^ \dagger a + \sigma_z/2 + 1/2$ appears non-trivially. 
defining
\begin{equation}
    \mathscr{N}_{\text{crit}} = \frac{\delta_{cq}^2}{4g^2}
\end{equation}
We perform an expansion in $\mathscr{N}/\mathscr{N}_{\text{crit}}$.
\begin{align}
    \Delta &= \sqrt{\delta_{cq}^2 + 4g^2 \mathscr{N}}\\
           &= \delta_{cq} \sqrt{1 + \frac{4g^2\mathscr{N}}{\delta_{cq}^2}}\\
           &= \delta_{cq} \sqrt{1 + \frac{\mathscr{N}}{\mathscr{N}_{\text{crit}}}}\\
           & \approx \delta \left(
             1
             + \frac{1}{2}\frac{\mathscr{N}}{\mathscr{N}_{\text{crit}}}
             + (1/8) \frac{\mathscr{N}^2}{\mathscr{N}^2_{\text{crit}}}
             \right)
\end{align}
where 
The hamiltonian becomes:
\begin{align}
    \mathscr{H} &= \omega_c a ^ \dagger a
    + (\omega_q/2) \sigma_z
    +  \xi/\sqrt{2} (a + a^\dagger) \cos(\omega_d t)\\
    &- \frac{4g^2}{\delta_{cq}}\left(a^\dagger a 
    +  \frac{\sigma_z}{2} + \frac{1}{2}\right)\sigma_z\\
    &- \frac{g^4}{\delta_{cq}^3}\Big\{
    \left(a^\dagger a\right)^2
    + \left(\frac{\sigma_z}{2}\right)^2
    + \left(\frac{1}{2}\right)^2\\
    &+ \frac{1}{2} \left(
                    a^\dagger a \sigma_z + \sigma_z a^\dagger a
                  \right)
    + a^\dagger a + \frac{\sigma_z}{2}
    \Big\} 
    \sigma_z
\end{align}
\subsubsection{First Order}
To first order in $\frac{\mathscr{N}}{\mathscr{N}_{\text{crit}}}$ the hamiltonian is:
\begin{align}
  \mathscr{H} &= \left(\omega_c
    - \frac{4g^2\sigma_z}{\delta_{cq}}\right) a ^ \dagger a
    + (\omega_q/2 - \frac{2g^2}{\delta_{cq}} \left(\sigma_z + 1)\right)\sigma_z\nonumber\\
    &+ \mathscr{H}_{drive}
\end{align}
In the dispersive limit, we can make the assumption that the qubit is unaffected by the dynamics, we make the approximation $\sigma_z = \pm 1$ - that the qubit is initialised either up or down, and remains in it's initial state for the system evolution.

We define $\chi_{JC} = \flatfrac{4g^2}{\delta_{cq}}$, and rescale energy levels by dropping constant terms.
\begin{equation}
    \mathscr{H} = \left(\omega_c \pm \chi_{JC}\right) a ^ \dagger a
    + \xi/\sqrt{2} ( a + a^\dagger ) \cos(\omega_d t)
\end{equation}
We are left with the bare cavity hamiltonian with a qubit-state dependent frequency shift
\subsubsection{Second Order in $\frac{\mathscr{N}}{\mathscr{N}_{\text{crit}}}$}
Keeping terms up to second order in the expansion of $\Delta$, we define shifted parameters
\begin{align*}
    \omega_c^{JC} &= \omega_c \pm \frac{4g^2}{\delta_{cq}}\\
    \mathscr{A}_{JC} &= \frac{g^4}{\delta_{cq}^3}
\end{align*}
in the second order hamiltonian. 
\begin{equation}
    \mathscr{H} = \omega_c^{JC} a ^ \dagger a
    + \mathscr{A}_{JC}\left(a^\dagger a\right)^2
    + \xi/\sqrt{2} ( a + a^\dagger ) \cos(\omega_d t)
\end{equation}
From \cite{Drummond1979} we have the Hamiltonian for a nonlinear medium in a cavity, equivalently, the quantum Duffing oscillator. 
\begin{equation}
    \mathscr{H} = \omega_c' a^\dagger a
    + \mathscr{A} : a ^ \dagger a a ^ \dagger a :
    + \xi'/\sqrt{2}(a+a^\dagger)\cos(\omega_d t)\label{duff}
\end{equation}
To second order, the dispersive Jaynes Cummings oscillator with the qubit frozen is a quantum Duffing/Kerr oscillator. 
From (crossref section generalised P) we have analytical expressions for the normally ordered moments - the cavity field, the correlation functions. 
Provided the second order behaviour persists in the non-perturbative limit, i.e. that this expansion is convergent, we would then expect a bunched photon transmission for a small range of detunings, followed by a broad region in which we have antibunched transmission.
We also would expect to see a dip in the transmission at a critical detuning, where there is quantum interference between two stable mean field states.
\begin{figure*}[pht]
  \centering
  \begin{minipage}{0.5\linewidth}
    \vspace*{-0.5cm}
    \includegraphics[width=1.35\linewidth]{dispersivebistability.pdf}
  \end{minipage}%
  \begin{minipage}{0.5\linewidth}
    \includegraphics[width=\linewidth]{scurve.pdf}  
  \end{minipage}
  \caption{Level lines of \cref{eq:sc_dispersive} (a) Contours of constant drive (b) Contours of constant detuning}
  \label{scurve}
\end{figure*}
\subsection{Semiclassical}
Following \cite{Bishop2010}, we build a semiclassical model.
Rewriting the hamiltonian in terms of the generalised coordinates $X = \frac{1}{\sqrt{2}}(\cre + \ann)$, $P = \frac{1}{\sqrt{2}}(\cre-\ann)$
\begin{align}
        \mathscr{H} &= \omega_c/1 (X^2 + P^2 + \sigma_z) + \xi X \cos(\omega_d t)\nonumber\\
                    & - \sigma_z /2 \sqrt{2g^2(X^2+P^2+\sigma_z) + \delta^2}
\end{align}
From Hamilton's equations for the (semi-)classical Hamiltonian
\begin{align}
        \frac{d\mathscr{H}}{dX} &= \frac{dP}{dt}\\
        \frac{d\mathscr{H}}{dP} &= -\frac{dX}{dt}
\end{align}
in the steady state, (which we access setting the second derivatives of the quadratures X \& P to zero) and solving for the amplitude $A = X^2 + P^2$, we find the amplitude self consistency equation.
\begin{equation}
        A^2 = \frac{\omega_d^2\xi}{\{\omega_d^2 - [\omega_c - \chi (A) ]^2 \}^2+ \kappa^2 \omega_d^2}
\end{equation}
where
\begin{equation}
        \chi(A) = \sigma_z \frac{g^2}{\sqrt{2g^2(A^2 + \sigma_z) + \delta^2}}
        \label{eq:sc_dispersive}
\end{equation}
inverting the equation and solving for $\xi$ as a function of $A^2$. 
The level curve of $A^2$ defines a manifold on the drive-detuning parameter space. 
The level set defines what is called a cusp catastrophe in the theory of dynamical systems.
Trajectories on this manifold from the point of view of catastrophe theory are considered in \cite{Agrawal1979} .  
The system exhibits hysteresis both in detuning and drive. 
We plot the contours of constant drive in \cref{fig:sc_dispersive}, and of constant detuning in \cref{scurve}.
These regions in the quantum case are washed out by switching induced by quantum fluctuations, and the amplitude in the quantum case can be qualitatively expected to 'average out' the bistability and lie between the two stable states on this plot.
Indeed, this effect is visible in \cref{fig:sc_dispersive}.
We see more interesting features here too, most notably a dip in the quantum amplitude. The nature of quantum probability means a probability distribution on the mean field curve will show interference effects between paths, here as amplitudes lower than the mean-field stable states.
\subsection{Bistable ``leaf"}
\begin{figure}[ht]
        \label{BistabilityLeaf}
        \includegraphics[width=\linewidth]{01:03:2016 - BishopBimodalityLeaf.png}
        \caption{Edges of the region of dispersive bistability}
\end{figure}
The drive opens and closes a region of dispersive bistability. By considering the derivatives of detuning with respect to drive, with $\xi$ as a parameter, we can demarcate this onset of bistability, and the draw the region in detuning/drive space in which bistability exists.
\subsection{Dispersive QND Readout}
\label{disp_QND_readout}
\begin{figure}[ht]
  \label{transmission}
  \centering
  \includegraphics[width=\linewidth]{transmission.pdf}
  \caption{qubit dependent transmission spectrum for dispersive system}
\end{figure}
The cavity resonance with the qubit dispersive shift reads $\omega_c - 4\flatfrac{g^2}{\delta_{cq}}\sigma_z$
For a qubit in the ground state i.e. $\sigma_z = -1$, the system transmission spectrum is peaked at $\omega_c + 4\flatfrac{g^2}{\delta_{cq}}$.
For a qubit in the excited state i.e. $\sigma_z = 1$, the system transmission is peaked at $\omega_c - 4\frac{g^2}{\delta_{cq}}$
It is clear that the state of the qubit affects a system observable, and thus that if the two transmission peaks can be resolved by experiment then the qubit state can be readout.
It remains to  implement a scheme for measuring the observable without destroying the qubit state. 
Such a scheme goes as follows, following \cite{Blais2004a}.
We assume $\flatfrac{g^2}{\kappa\delta_{cq}} >1$ that is, that the peaks are resolvable.
Then, irradiating the cavity at either of the shifted frequencies, the transmission will be near unity if the qubit is in the state that shifts the resonance to that frequency, and near zero otherwise (see \cref{transmission}).
Alternatively, we irradiate the cavity coherently at the unshifted resonance $\omega_c$. 
Given that the qubit is initially in the superposition $\alpha \ket{1} + \beta \ket{0}$, the transmitted coherent field will evolve into the entangled pair $\alpha \ket{1, \chi} + \beta \ket{0, -\chi}$, where $\ket{\chi}$ are appropriate coherent states for the transmitted field. 
It is clear that projectively measuring the transmitted field constitutes an equivalent projective measurement on the qubit.
It can be shown \cite{Blais2004a} that the second scheme is more suitable (has higher fidelity) when the peaks are more easily resolvable $\flatfrac{g^2}{\kappa\delta_{cd}} \ll 1$.
\subsection{Coherent Control}
Another of the key requirements of a quantum computing implementation is the possibility of coherent control of the qubit state. 
Here, again, the dispersive regime is important. 
Driving the qubit at its resonance will not affect the combined system i.e. does not constitute a measurement in the previous sense, because of the magnitude of $\delta_{cq}$.
Provided we have the necessary precision in the laser frequency, qubit rotations can be performed with very high fidelity \cite{Blais2004a}.
