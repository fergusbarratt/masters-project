\section{The Jayes-Cumming Hamiltonian}
In the Jaynes-Cummings model, a single cavity mode interacts with a two level system. 

We start with the total Hamiltonian
\begin{equation}
	\mathscr{H} = \mathscr{H}_A + \mathscr{H}_F +\mathscr{V}_{int}
\end{equation}
\subsection{The Dipole Approximation}
In full generality, the field will interact with all the higher order dipole moments of the atom.
However, given that the spatial variation of typical optical fields is minimal on the order of an atom ($\sim$ \AA), the interaction hamiltonian can be approximated by the interaction of the electric field only with the electric dipole moment of the atom
\begin{equation}
	\hat{\mathscr{V}}_{int} = -\vec{\hat{d}} \cdot \vec{\hat{E}}
\end{equation}
The quantized electromagnetic multimode vector potential in a medium can be expressed in the following way \cite[271--273]{Novotny2006}(we consider only a countable number of field modes, since we will put the field in a cavity)
\begin{equation}
	\hat{A} = \sum_{\vec{k}, \mu} \sqrt{\frac{\hbar}{2\omega_{\vec{k}} V \varepsilon_0}} (\vec{u}_{\vec{k}} \ann + \vec{u}^*_{\vec{k}} \cre )
\end{equation}
with $\vec{u}_{\vec{k}}$ orthogonal normal modes satisfying the wave equation:
\begin{equation}
	\nabla \times \nabla \times \vec{u}_{\vec{k}} = \frac{\omega_{\vec{k}}^2}{c^2}  \vec{u}_{\vec{k}}
\end{equation}
We consider only coupling to a single mode of the cavity field:
\begin{equation}
	\hat{A} =  \sqrt{\frac{\hbar}{2\omega_{\vec{k}} V \varepsilon_0}} (\vec{u}_{\vec{k}} \ann + \vec{u}^*_{\vec{k}} \cre )
\end{equation}
from which $\vec{E}$ for the mode is easily derived since (in the radiation gauge)
\begin{equation}
	\vec{\hat{E}} = -\frac{\partial}{\partial t}\vec{\hat{A}}
\end{equation}
\subsection{Two Level Approximation}
$\vec{\hat{d}} = e \vec{\hat{r}}$ can be expanded in the space of atomic levels by resolving unity on each side
\begin{equation}
	\vec{\hat{d}} = e \sum_{a, b} | a \rangle \langle a | \vec{\hat{r} }| b \rangle \langle b |
\end{equation}
since we consider only two levels, the sum can be truncated:
\begin{equation}
	\vec{\hat{d}} = e\{| 1 \rangle \langle 1|\vec{\hat{r}} | 1 \rangle \langle 1 | + | 1 \rangle \langle 1|\vec{\hat{r}} | 2 \rangle \langle 2 | + | 2 \rangle \langle 2|\vec{\hat{r}} | 1 \rangle \langle 1 | + | 2 \rangle \langle 2|\vec{\hat{r}} | 2 \rangle \langle 2 | \}
\end{equation}
we assume the atom has no permanent dipole moment and neglect terms of the form $| 1 \rangle \langle 1|\vec{\hat{r}} | 1 \rangle \langle 1 |$.
Expressing in terms of the atomic creation and annihilation operators
\begin{equation}
	\vec{\hat{d}} = (m\sigma_+ + m^* \sigma_-)
\end{equation}
where $m, m^*$ are the electric dipole matrix elements
\subsection{The Rotating Wave Approximation}
The free field hamiltonian, neglecting the zero point energy has the form
\begin{equation}
	\ham_F =  \hbar \omega \cre \ann
\end{equation}
The atom energy, neglecting centre of mass motion and considering only population inversion, in terms of the Pauli z matrix:
\begin{equation}
	\ham = \frac {1} {2} \hbar \omega_q \sigma_z
\end{equation}
where $\omega_q$ is the frequency of the bare atomic transition
Assuming sinusoidal mode functions with polarisation $\varepsilon_\Omega$, and incorporating constants into new dipole matrix elements $g = \frac{m \varepsilon_\Omega sin(Kz)} {2 \hbar}$, the total hamiltonian takes the form
\begin{equation}
	\ham = \hbar \omega \cre \hat{a} +\frac{1}{2} \hbar \omega_q \sigma_Z + \hbar (\ann +\cre)(g\atann+g^*\atcre)
\end{equation}
expanding the interaction term
\begin{align}
  \hat{\mathscr{V}}_{int} &= \hbar (\ann +\cre)(g\atann+g^*\atcre) \\
                          &=  \hbar (g \ann \atann + g^* \ann \atcre + g \cre \atcre +g^* \cre\atann)
\end{align}
moving to an interaction picture rotating at the transition frequency $\omega_q$ it can be seen that terms of the form $\atann \cre $ counterrotate with frequency $\omega + \omega_L$ and terms of the form $ \atann \ann$ co rotate with frequency given by the detuning $\omega-\omega_L$.
The rotating wave approximation is the assumption that the counterrotating terms will quickly average to zero on the system timescale, and thus can be dropped in the expansion of the interaction hamiltonian.
This approximation has broad application, especially at optical frequencies \cite, but it has its limits \cite{Dolce2006} \cite{Milonni1995} \cite{Biswas1990}.
\subsection{The Jaynes-Cumming Hamiltonian}
Applying all of the above leads to a hamiltonian $\ham_{JC}$ of the form:
\begin{equation}
	\ham_{JC} = \hbar \omega \cre \hat{a} +\frac{1}{2} \hat{\sigma}_Z \omega_q + \hbar g (\cre \atann + \hat{a} \atcre)
\end{equation}
\subsection{Solving the Jaynes Cumming system in the absence of drive}
\subsubsection{Bare states}
We first move to an interaction picture rotating with the bare system, partitioning the Hamiltonian
\begin{align}
	\ham_I &= \hbar \omega \hat{N}_e + \hbar(\frac{\omega_q}{2}-\omega)\hat{P}_E \\
	\ham_{II} &= -\hbar \Delta + \hbar g (\cre \atann + \hat{a} \atcre) \\
	\ham_{JC} &= \ham_I+\ham_{II}
\end{align}
with $\Delta = \omega-\omega_q$,   $\hat{N}_e = \ket{2} \bra{2} + \ket{1} \bra{2} $ the (conserved) electron number and $\hat{P}_E = \cre\ann + \ket{2}\bra{2} $ the (conserved) excitation number.

We consider evolution of the Jaynes-Cumming system in the absence of decay, dephasing and detuning, in the bare state basis.
Since the field only couples successive levels of the combined atom-field system, the state of the closed system can be described in the restricted two level Hilbert space
\begin{equation}
	\ket{\psi(t)} = C_1(t) \kettens{1}{g} + C_2(t) \kettens{0}{e}
\end{equation}
where $\ket{\psi}$ is understood to be an element of the atom-field Hilbert space.
We solve the interaction schrodinger equation to determine the time dependent coefficients $C_1(t)$ and $C_2(t)$
\begin{equation}
	\schro{\psi(t)}{\ham_{II}}
\end{equation}
and get two coupled differential equations for the level amplitudes:
\begin{align}
	\frac{d C_2(t)}{dt} &= -i g \sqrt{n+1}C_1(t)\\
	\frac{d C_1(t)}{dt} &= -i g \sqrt{n+1}C_2(t)
\end{align}
which are solved by
\begin{align}
  C_1(t) &= \cos(g\sqrt{n+1}t)\\
  C_2(t) &= \sin(g\sqrt{n+1}t)
\end{align}
\subsubsection{Dressed States}
The uninteracting hamiltonian 
\begin{equation}
  \mathscr{H}_{\text{bare}} = \hbar \omega_c + \hbar \omega_q \flatfrac{\sigma_z}{2}
\end{equation}
satisfies 
\begin{align}
  \mathscr{H} \ket{0, n} &= \hbar \left\{\frac{1}{2} \omega_q + \omega_c \right\} \ket{0, n}\\
  \mathscr{H} \ket{1, n} &= \hbar \left\{-\frac{1}{2} \omega_q + \omega_c \right\} \ket{1, n}
\end{align}
where $\ket{q, n} = \kettens{q}{n}$, $\{\ket{n}\}$ field fock states. 
The interacting hamiltonian
\begin{align}
  \mathscr{H}_I = \hbar g \left(\ann \atcre + \atann \cre\right)
\end{align}
couples only within pairs $\{\ket{0, n+1},\ \ket{1, n}\}$
We can consider the hamiltonian as acting only in the two dimensional space spanned by these states. 
In the matrix representation we have \cite{Meystre2007}
\begin{align*}
  \mathscr{H} &= \mathscr{H}_{\text{bare}} + \mathscr{H}_I \\
              &= \hbar \left(n+\frac{1}{2}\right) \omega_c 
  \begin{pmatrix}
    \mqty{1 & 0 \\ 0 & 1}
  \end{pmatrix}
              + \hbar / 2
  \begin{pmatrix}
    \mqty{\delta_{cq} & 2g\sqrt{n+1} \\ 2g\sqrt{n+1} & -\delta_{cq}}
  \end{pmatrix}
\end{align*}
The first matrix is obviously diagonal. 
The second is easily diagonalisable, with eigenvalues
\begin{align}
  \mathcal{E}_{-, n} &= \mathcal{E}_{\text{bare}} - \frac{1}{2} \hbar \sqrt{\delta_{cq}^2 + 4g^2 \left(n+1\right)}\\
  \mathcal{E}_{+, n} &= \mathcal{E}_{\text{bare}} + \frac{1}{2} \hbar \sqrt{\delta_{cq}^2 + 4g^2 \left(n+1 \right)}
\end{align}
where
\begin{equation}
  \mathcal{E}_{\text{bare}} = \hbar\left(n + \frac{1}{2}\right) \omega_c
\end{equation}
and eigenvectors
\begin{align}
  \ket{+}_n &= \cos(\theta) \ket{0, n+1} - \sin(\theta) \ket{1, n} \\
  \ket{-}_n &= \sin(\theta) \ket{0, n+1} + \cos(\theta) \ket{1, n} 
\end{align}
where
\begin{equation}
  \tan{2\theta} = -\frac{2g\sqrt{n+1}}{\delta_{qd}}
\end{equation}
These eigenstates are known as the \emph{dressed states}.
The atomic states are \emph{dressed} by the cavity field, and the energy levels are split by an amount $\hbar \sqrt{\delta_{cq}^2 + 4g^2\left(n+1\right)}$.
This is known as \emph{anti} or \emph{avoided} crossing.
We have the following relation between the dressed and bare level amplitudes
\begin{equation}
  \begin{pmatrix}
    \mqty{C_+(t) \\ C_-(t)}
  \end{pmatrix}
  =
  \mathcal{T}
  \begin{pmatrix}
    \mqty{C_{1, n} \\ C_{0, n+1}}
  \end{pmatrix}
\end{equation}
where
\begin{equation}
  \mathcal{T} = 
  \begin{pmatrix}
    \mqty{\cos(\theta) & -\sin(\theta) \\ \sin(\theta) & \cos(\theta) }
  \end{pmatrix}
\end{equation}
is a the unitary rotation matrix from one basis to the other.
From the time evolution operator representation of the schrodinger equation, after resolving unity with the dressed states we have for the bare state probability amplitudes. 
\begin{align*}
  \ket{\psi(t)} &= \exp{-i\mathscr{H}t/\hbar} \ket{\psi(0)}\\
                &= \sum_{n=0}^\infty \sum_{\pm} \exp{-i\mathscr{E}_{\pm, n}t/\hbar} \ket{\pm}_{n \ n}\bra{\pm}\ket{\psi(0)}\\
                &= \exp{\frac{1}{2} i \sqrt{\delta_{cq}^2 + 4g^2 \left(n+1\right)} t/\hbar} C_+(0) \ket{+}_n\\
                &+ \exp{-\frac{1}{2} i \sqrt{\delta_{cq}^2 + 4g^2 \left(n+1\right)}t/\hbar} C_-(0) \ket{-}_n
\end{align*}
\begin{widetext}
\begin{align*}
  \implies \begin{pmatrix}
              \mqty{C_{1, n}(t) \\ C_{0, n+1}(t)}
            \end{pmatrix} 
            &=\mathcal{T}^{-1} 
               \begin{pmatrix}
                 \mqty{\exp{\frac{i}{2}\sqrt{\delta_{cq}+4g^2(n+1)}t/\hbar} & 0 \\ 
                 0 & \exp{-\frac{i}{2}\sqrt{\delta_{cq}+4g^2(n+1)}t/\hbar} }
               \end{pmatrix}
   \mathcal{T}
   \begin{pmatrix}
      \mqty{C_{1, n}(0)\\C_{0, n+1}(0)}
   \end{pmatrix}\\
   &=\begin{pmatrix}
   \cos(\frac{1}{2}\Theta t) - \frac{i\delta_{cq}}{\Theta} \sin(\frac{\Theta}{2}t)& -\frac{2ig\sqrt{n+1}}{\Theta}\sin(\frac{\Theta}{2}t)\\
   -\frac{2ig\sqrt{n+1}}{\Theta} \sin(\frac{\Theta}{2}t) & \cos(\frac{\Theta}{2}t) + \frac{i\delta_{cq}}{\Theta}\sin(\frac{\Theta}{2}t) 
     \end{pmatrix}
     \begin{pmatrix}
      C_{1, n}(0)\\
      C_{0, n+1}(0)
     \end{pmatrix}
\end{align*}
\end{widetext}
with 
\begin{equation*}
  \Theta = \sqrt{\delta_{qd}^2 + 4g^2 (n+1)}
\end{equation*}
with $\Theta \rightarrow 2g\sqrt{n+1}$ as $\delta_{cq} \rightarrow 0$.
We have a simple expression for the coefficients of the solution in the bare basis 
\begin{equation}
  \ket{\psi(t)} = C_{1, n}(t) \ket{1, n} + C_{0, n+1}(t) \ket{0, n+1}
\end{equation}
For arbitary detuning and initial conditions. 

\subsection{Initial Conditions}
\subsubsection{Fock State, atom initially excited}
The solution is 
\begin{align*}
  \ket{\psi(t)} &= \left(\cos(\frac{\Theta}{2}t)-\frac{i\delta_{cq}}{\Theta}\sin(\frac{\Theta}{2}t) \right)\ket{1, n} \\
                &-\frac{2ig\sqrt{n+1}}{\Theta}\sin(\frac{\Theta}{2}t)\ket{0, n+1}
\end{align*}
It is clear that for $\delta_{cq} \rightarrow 0$ the solution returns to that we derived in the previous section. 
\begin{figure}[h]
  \includegraphics[width=\linewidth]{rabi_resonance.pdf}
  \caption{Qubit excited $\ket{1, n}$ and ground $\ket{0, n+1}$ state probability.}
  \label{rabi_resonance}
\end{figure}
\begin{figure}[h]
  \includegraphics[width=\linewidth]{rabi_detuned.pdf}
  \caption{Qubit excited and ground state probability as in \cref{rabi_resonance}, now with non-zero cavity-qubit detuning, and at higher frequency.}
  \label{rabi_detuning}
\end{figure}
The excitation probability passes coherently from the qubit to the field with frequency given by the coupling strength multiplied by the square root of the excitation number of the field.
On the Bloch sphere, the qubit vector is rotating around the sphere origin, hitting both the poles in one sweep. 

When we add cavity-qubit detuning to the system, the rotation angle on the bloch sphere increases from zero, and the vector misses the poles. 
The qubit now never passes all of its excitation probability to the field, since the coupled systems no longer have the same resonant frequency. 
\subsection{Photon Blockade}
We add a driving field to the Jaynes-Cummings hamiltonian. It is added coherently i.e.\ the system evolves unitarily with the drive, rather than the more general incoherent case (for example as in \cite{Xu2014}), and the Hamiltonian reads:
\begin{align}
    \ham_{JC} &= \hbar \omega_d \cre \ann + \hbar \omega_q \atann \atcre +\hbar g (\cre \atann+ \hat{a} \atcre)\\
    & + \hbar \xi (\cre e^{i\omega_d t} + \ann e^{-i\omega_d t})
\end{align}
We also consider the effect of dissipation via two collapse operators, the spontaneous decay of the atom via $\atann$, with strength $\gamma$ and the decay of the cavity field via $\ann$, with strength $\kappa$. The total master equation reads:
\begin{equation}
  \dot{\rho} = \frac{1}{i\hbar}[\ham_{JC}, \rho] + \mathscr{L}_\kappa[\rho] + \mathscr{L}_\gamma[\rho]
\end{equation}
where $\mathscr{L}$ represents the Lindblad dissipator for each collapse parameter. 
We ignore pure dephasing.

We first move to a frame rotating at the drive frequency, giving a hamiltonian:
\begin{equation}
  \ham = \delta_{qd} \cre \ann + \delta_{cd} \atcre \atann + \hbar g (\cre \atann + \ann \atcre) + \hbar (\ann + \cre)
\end{equation}
where the explicit time dependence has been removed.
With the cavity field and the qubit on resonance ($\delta_{cd} = \Delta$), and for now neglecting the drive, the Jaynes Cummings Hamiltonian can be written.
\begin{equation}
  \ham_{JC} = \hbar \Delta (\cre \ann + \atann \atcre) +\hbar g (\cre \atann + \hat{a} \atcre)
\end{equation}
Diagonalising yields the balanced dressed states.
\begin{align}
  \ket{E_{n, U}} & = \frac{1}{\sqrt{2}} (\kettens{n}{0}+\kettens{n-1}{1}) \\
  \ket{E_{n, L}} & = \frac{1}{\sqrt{2}} (\kettens{n}{0}-\kettens{n-1}{1})
\end{align}
in the tensor product of the field fock space and the atomic eigenspace spanned by the bare eigenstates $\ket{1}, \ket{0}$. These dressed eigenstates are superpositions of the bare states $\kettens{n}{-}$ and $\kettens{n-1}{+}$ and are balanced only in the resonant case. The eigenenergies are
\begin{align}
  E_{n, U} &= n \hbar \omega_0 + \sqrt{n} \hbar g \\
  E_{n, L} &= n \hbar \omega_0 - \sqrt{n} \hbar g
\end{align}
in which the Rabi splitting between the upper and lower dressed states is clear. In the absence of coupling to a dressing field $(g=0)$ the Jaynes-Cumming energies form a degenerate harmonic ladder; it is clear that the qubit coupling induces an anharmonicity via the characteristic $\sqrt{n}$ Rabi splitting.

We now consider the effect of an external drive tuned to the $\ket{G} \rightarrow \ket{E_{1, U/L}}$ transition\footnote{Drive frequencies at multiphoton resonances induce the same effect} (where $\ket{G}$ is the coincident dressed ground state $\ket{E_{0, -}} = \kettens{0}{-}$) with frequency $\omega_D = \hbar \omega_0 \pm \hbar g$.
The $\kettens{1}{U/L} \rightarrow \kettens{2}{U/L}$ step of the Jaynes Cummings ladder is now detuned from the drive by $E_{2, U/L} - E_{1, U/L} - \hbar \omega_D =  \mp(2-\sqrt{2}) \hbar g$. Thus for sufficiently large g and sufficiently small linewidth, the upper steps of the ladder are inaccessible, and the Jaynes Cumming system is opaque to further photon absorption until the photon is reemitted from the cavity through some loss process. This is the photon blockade effect \cite{Birnbaum2005}.
