\section{The Quantum Master Equation}
The quantum master equation is a generalisation of the classical master equation, which describes the time evolution of a system confined to a set of states, to the case where quantum correlations between different states are important. 

In the nonrelativistic theory of quantum mechanics the time evolution of the pure state state vector is described by the \emph{Schr\"odinger equation}:
\begin{equation}
        i\hbar \pdv{t} |\psi(t)\rangle = \ham(t) |\psi(t) \rangle
        \label{eq:schrodingerequation}
\end{equation}
with $\ham$ the Hamiltonian of the system.
We set $\hbar$ to 1 going forwards. 
Assuming a time independent Hamiltonian, the time dependence in the above can also be represented in terms of a time evolution operator (generated by the Hamiltonian)
\begin{equation}
        |\psi(t) \rangle = U(t, 0) | \psi(0) \rangle 
        \label{eq:timeevolutionoperator}
\end{equation}
by substitution of \cref{eq:timeevolutionoperator} into \ref{eq:schrodingerequation}, it is easy to see that $U^{\dagger}U = UU^\dagger = I$ provided $\ham$ is hermitian i.e. U represents unitary time evolution of the system

For a mixed state of a closed system, an equation of motion for the density matrix
\begin{equation}
	\dens (t) = \sum_\alpha w_{\alpha} |\psi_\alpha (t) \rangle \langle \psi_\alpha (t) |
\end{equation}
can be found by propagating each normalised $| \psi_\alpha (t) \rangle$ with the time evolution operator, the result of which is more concisely expressed
\begin{equation}
	\dens (t) = U(t, 0)\dens (0) U^\dagger (t, 0)
\end{equation}
which, when differentiated with respect to time yields the \emph{Liouville-Von Neumann} equation
\begin{equation}
	\frac{d}{dt}\dens(t) = i [\ham, \dens (t) ]
\end{equation}
For open quantum systems, i.e. those coupled to other systems, the dynamics are in general not unitary. 
Mathematically, we partition a larger, unitarily evolved system into system and environment, and perform a partial trace over the environment degrees of freedom.

We start with the Hamiltonian for a closed, potentially mixed quantum system and partition it into components representing: a subsystem constituting the totality of the interesting dynamics: the \emph{system}, a subsystem representing the dynamics of the remaining degrees of freedom the \emph{environment} or \emph{reservoir}, and a component representing the interaction between system and environment (there is an associated partitioning of the Hilbert space into the tensor product of system and environment Hilbert spaces, and in the equation below it is understood that operators with the subscript S operate only on the system degrees of freedom i.e. exist in the system Hilbert space and represent identity in the environment Hilbert space, and etc.)
\begin{equation}
	\ham = \ham_S +\ham_R + \ham_I
\end{equation}
The partial trace operation is an operator-valued function that takes a operator on a larger Hilbert space and discards its action on all but a smaller Hilbert space, colloquially ``tracing over'' the degrees of freedom of the discarded subsystem to leave the operator defined on the smaller subsystem.
It is defined 
\begin{equation}
  \tr{\hat{O}} = \sum_\sigma\ev{\hat{O}}{\sigma}
  \label{trace}
\end{equation}
where $\{\ket{\sigma}\}_{\sigma \in I}$ is a basis of the discarded space.
Unitarily evolving the large system, and performing the partial trace over the environment.
\begin{equation}
	\dens_S (t) = tr_E \{U(t, 0) \dens (0) U^\dagger (t, 0) \}
\end{equation}
with $\dens_S = tr_E\{\dens \}$ we arrive at the most general form for evolution of a subsystem, also known as the reduced master equation
\begin{equation}
	\frac{d}{dt} \rho_S (t) = -itr_E\{[\ham(t), \dens_S(t)]\}
        \label{general}
\end{equation}
Which is presented here in full generality. We now consider solvable approximations, and alternative conceptualisations.
\subsection{Quantum Measurements}
One general theory of non unitary evolution is furnished by the generalised theory of quantum measurement \cite{Wiseman2010a}
\subsubsection{Von-Neumann projective measurements}
The standard formulations of quantum mechanics considers measurements via the Von-Neumann-L\"uders Projection Postulate \cite{Dirac1927}
\newtheorem{definition}{Definition}
\begin{definition}
    The result of a measurement of operator $A$ with spectral decomposition $A = \sum_\lambda \lambda \hat{\Pi}_\lambda $, (for $A$ with discrete, non-degenerate spectrum) is an eigenvalue $lambda$, and state of the system conditioned on receiving such a value is given by
    \begin{equation}
        \rho_\lambda = \frac{\hat{\Pi}_\lambda \rho \hat{\Pi}_\lambda}{Tr[\rho\hat{\Pi}_\lambda]}
    \end{equation}
\end{definition}
The quantum analog of the Bayesian rule for the update of probabilities.
In general, such a description does not represent the measurements that experimentalists perform on quantum systems, and nor does it consider the coupling of every real quantum system to its environment.
More generally, we deal with the formalism of Operators \& Effects.
We consider a probe (or pointer/meter/apparatus) system through which we measure the system of interest.
Let the initial state of the combined probe-system state be a pure product of probe-system kets
\begin{equation}
  | \Psi (t) \rangle = | \theta (t) \rangle \otimes | \psi (t) \rangle
\end{equation}
We let the coupled system evolve unitarily for some time $t_1$, entangling the system and the probe.
We now perform a Von-Neumann projective measurement on the probe, through some projector $\pi_\lambda = \Pi_\lambda \otimes \hat{I}$ acting only on the probe space over some second period of time $t_2$ with $t_1 + t_2 = T$.
From the projection postulate, we have as conditional final state:
\begin{equation}
    | \Psi_\lambda (t+T) \rangle = \frac{\pi_\lambda U(t_1) |\Psi(t)\rangle}{P}
\end{equation}
where P is the probability that we obtain result $\lambda$ when measuring the probe.
Measuring the entangled state projectively disentangles the probe and the system and represents an operation on the system space given by
\begin{equation}
\Psi_\lambda = | \lambda \rangle \hat{M}_\lambda | \psi(t) \rangle / P
\end{equation}
where we implicitly consider nondegenerate spectra by decomposing the projector $\Pi_\lambda = |\lambda \rangle \langle \lambda |$, and where
\begin{equation}
     \hat{M}_\lambda = \langle \lambda | U(t_1) | \theta(t) \rangle
\end{equation}
is the \emph{measurement operator}, which now acts only only on the system Hilbert space.
The normalisation factor P is expressed in terms of these operators
\begin{equation}
     P^2 = \langle \Psi (t) | \hat{M}_\lambda^\dagger \hat{M}_\lambda | \Psi (t) \rangle
\end{equation}
 Our description of the measurement process has been so far effectively independendent of the meter.
We can thus abstract over the particular form of the probe state.
The state of the system following the generalised measurement process is given through the measurement operators $\hat{M}_\lambda$ as follows
\begin{equation}
 | \psi_\lambda (t + T) \rangle = \hat{M}_\lambda | \psi(t) \rangle /P
\end{equation}
 and the probability that the system takes some state is given through the \emph{effects} (probability operators) $ \mathscr{E}_\lambda = \hat{M}_\lambda^\dagger \hat{M}_\lambda $ as
\begin{equation}
     P^2 = \mathscr{P}_\lambda = \langle \psi_\lambda (t) | \mathscr{E}_\lambda | \psi_\lambda \rangle
\end{equation}
 the set $\{\mathscr{E}_\lambda | \sum_\lambda \mathscr{E}_\lambda = \hat{I} \} $ is known as a POVM: a \emph{Positive Operator Valued Measure} \footnote{Positivity is clear since the operators $\hat{M}_\lambda$ are positive.
Positivity, and that the operators resolve unity constitute the only restrictions on these operators}
We now consider a finite sequence of quantum measurements of this kind.
From the above, after some given timestep T, (we now consider propagating mixed states rather than pure states)
\begin{equation}
     \rho_\lambda (T) = \mathscr{F}_T[\hat{M}_\lambda] \rho(0) / \mathscr{P}_\lambda
\end{equation}
With $\mathscr{F}_T[\hat{M}_\lambda] ( \hat{O} ) =  \hat{M}_\lambda \hat{O} \hat{M}_\lambda ^ \dagger$ a \emph{superoperator}
This superoperator is an \emph{operation} for the value $\lambda$.
We consider the superoperators that represent physical time evolution.
\subsubsection{Physical Dynamical Maps}
A dynamical map on the space of density operators takes a quantum system at a time t to the same system at a time t'.
For consistency, we require that it meet certain criteria to be physical.
\begin{enumerate}
    \item \emph{Linearity}. We require that moving a system forwards in time respects superpositions 
        \begin{equation}
            \mathscr{F}_T[\alpha\rho(t) + \beta\rho'(t')] = \alpha \mathscr{F}_T[\rho(t) + \beta \mathscr{F}_T[\rho'(t')]
        \end{equation}
    \item \emph{Complete Positivity}.  We require that density matrices remain positive for all time.
        More strictly, we require that the operation of propagating some subsystem leaves the density matrix of the supersystem positive.
        \begin{align}
            p & \in  spectrum \{\mathscr{F}_T[\rho(t)]\} \geq  0\quad \forall p, T, t\\
            p & \in  spectrum \{\mathscr{F}_T \otimes \hat{I} [\rho_A(t) \otimes\rho_B(t) ] \} \geq 0 \quad \forall p, T, t
        \end{align}
        where the second inequality is true for all possible partitionings of the system into subsystems.
    \item \emph{Trace-Preserving}. We require that the trace of the density matrix is invariant under time propagation
        \begin{equation}
            tr\{ \mathscr{F}_T[\rho_t] \} = tr \{ \rho_t \} \quad \forall t, T
        \end{equation}
\end{enumerate}
It is a fundamental theorem of Kraus \cite{Kraus1983} that any such superoperator can be represented \cite{Nielsen2010}
\begin{equation}
    \mathscr{F}_T[\rho] = \sum_k \mathscr{E}_k^\dagger \rho \mathscr{E}_k
\end{equation}
where $\mathscr{E}_k $ are known as Kraus Operators, but are exactly the measurement operators $\mathscr{E}_\lambda$ derived above.
This decomposition is known as the Operator-Sum representation \cite{Nielsen2010}, and operators satisfying the above conditions are CPTP (Completely Positive, Trace-Preserving) maps.
The problem of describing the time evolution of a density operator reduces to finding the Kraus operators that represent the superoperator that generates the system time evolution.

This process admits a clear interpretation.
The nonunitary time evolution of a system corresponds to a countable series of measurement steps, where the system is non-orthogonally projected at each time step by the POVM consisting of the Kraus operators that give its time evolution.

\subsection{Quantum Dynamical Semigroups}
Markowian processes are colloquially ``memoryless'', a classical example being Brownian Motion.
Whilst the operator-sum representation can represent a more general time-evolution, most analytically solvable situations invoke what is known as the markow approximation.
Here we derive the master equation in the Markow approximation via the theory of Dynamical Semigroups, and in a microscopic formulation

\subsubsection{Markow Processes}
Classical Markow processes are characterised by the invocation of the so-called semigroup property of the Chapman-Kolmogorov equation.
The Markow transition probabilities $T(x, t|x', t')$ form a one parameter semigroup in t, because of the composition property $T(x, t|x, t')T(x, t'|x, t``) = T(x, t|,x, t``)$ which results from the conditions that memorylessness puts on their form.
This memorylessness is also responsible for this being a \emph{semi}group, since there does not in general exist an inverse for each element-there exist irreversible Markow processes.

This semigroup is in fact a "contracting" semigroup, in that the norm of a propagated probability is less than or equal to the probability.

The generalisation of Markow processes to quantum mechanics analagously leads to the theory of \emph{quantum dynamical semigroups}

\subsubsection{Lindblad Form}
The most general form \cite[119--122]{Breuer2002} for the reduced master equation in the Markow approximation is known as the \emph{Lindblad Form}
\begin{align}
        \frac{d}{dt} \rho_S (t) =& -i[\ham (t), \dens (t)\notag]\\
                                 & + \sum_{k=1}^{N^2-1} \gamma_k \{ \hat{A}_k \dens_S \hat{A}_k -\frac{1}{2}  \hat{A}_k \dens_S \hat{A}_k -\frac{1}{2} \dens_S \hat{A}_k \hat{A}_k \}\\
        = & -i[\ham, \dens(t)]+ \sum_k \mathscr{L}_{\gamma_{k}} [\rho]
\end{align}
with ${\{A_k\}}_{k \in I}$ some indexed complete orthonormal basis of operators on the system Hilbert space

\subsection{The Master Equation in Quantum Optics}
We start with the most general form for the reduced master equation \cref{general}
\begin{equation}
	\frac{d}{dt} \rho_S (t) = -itr_E\{[\ham(t), \dens_S(t)]\}
\end{equation}
with the hamiltonian of the form 
\begin{equation}
  \ham = \ham_S + \ham_R + \ham_{I}
\end{equation}
We assume that the interaction hamiltonian will take the form of a bilinear product of system and reservoir operators\footnote{This form effectively represents decay via one channel. For decay via multiple channels, the derivation proceeds identically with a sum of reservoir operators of this form, provided that the different reservoir operators commute.}
\begin{equation}
  \ham_{I} = \hat{S}\hat{R}^\dagger (t)+ \hat{R}(t)\hat{S}^\dagger
\end{equation}
We have moved to the interaction picture, and expressed the time dependence of the operator only in the reservoir components.
We make three assumptions about the form of the reservoir, collectively termed the \emph{Born-Markow Approximation}
\begin{enumerate}
  \item The reservoir is unaffected by the dynamics of the system.
    \begin{itemize}
      \item We take the reservoir state to be time independent.
    \end{itemize}
  \item The reservoir has a short correlation time (equivalently, a broad bandwidth).
    \begin{itemize}
      \item For $\tau_s$ a timescale characterising system dynamics, and $\tau_c$ the correlation time of the reservoir, we require $\tau_s \gg \tau_c$
      \item The dynamics of the reservoir are Markowian.
    \end{itemize}
  \item The system and reservoir remain effectively unentangled as far as the system dynamics are concerned.
    \begin{itemize}
      \item Specifically, we assume 
        \begin{align*}
          &\dens(t) = \rho_S(t)  \otimes \rho_R(t) + \rho_{\text{er}}\quad \forall t,\\
          &\\
          &\text{where}\\
          &\rho_{er} \rightarrow 0 \text{ as } \frac{\Delta t}{\tau_s} \rightarrow 0
        \end{align*}
      \item This is an interpretation of assumptions (1) and (2)
    \end{itemize}
\end{enumerate}
We perform a coarse graining of the timescales, and discard dynamics on scales $<\Delta t$, where $\tau_c \ll \Delta t \ll \tau_s$
Formally integrating \cref{general}
\begin{equation}
  \dens(t+\Delta t) - \dens(t) = -i \int_t^{t+\Delta t} \dd{t'} \comm{\ham_{I}(t')}{\dens(t')}
  \label{integrodifferential}
\end{equation}
We expand $\dens(t')$ around $t$, since from the limits on the integral $t-t'<\Delta t$
\begin{equation}
  \dens(t') \approx \dens(t) - i\int^{t'}_t\dd{t''}\comm{\ham_I(t)}{\dens(t'')}
\end{equation}
and insert into \cref{integrodifferential}
\begin{align*}
  \dd{\dens} &= -i\int_t^{t+\Delta t} \dd{t'} \comm{\ham_I(t')}{\dens(t) - i\int^{t'}_t\dd{t''}\comm{\ham_I(t)}{\dens(t'')}}\\
  \dd{\dens} &= -i\int_t^{t+\Delta t} \dd{t'} \comm{\ham_I(t')}{\dens(t)}\\
             &+   \int_t^{t+\Delta t} \int_t^{t'} \dd{t'}\dd{t''} \comm{\ham_I(t')}{\comm{\ham_I(t'')}{\dens(t'')}}\\
             &= \mathscr{I}_1 + \mathscr{I}_2
\end{align*}
We use the separability assumption, and perform the environment partial trace to simplify $\mathscr{I}_1$
\begin{align*}
  \mathscr{I}_1&= -i \int_t^{t+\Delta t} \dd{t'} \tr_E \comm{\hat{S}\hat{R}^\dagger (t')+\hat{R}(t')\hat{S}^\dagger}{\dens_S(t) \otimes \dens_R(t)}\\
               &= -i \int_t^{t+\Delta t} \dd{t'} \Bigg\{\comm{\hat{S}}{\rho_S(t)} \otimes \ev{\hat{R}^\dagger (t')} \\
               &\qquad \qquad \qquad \ \ - \comm{\hat{S}^\dagger}{\rho_S(t)} \otimes \ev{\hat{R}(t')}\Bigg\}\\
               &=-i \int^{t+\Delta t}_t \dd{t'} \comm{\ham_{av}(t')}{\dens_S(t)}
\end{align*}
where
\begin{equation}
  \ham_{av} = \hat{S} \ev{\hat{R}^\dagger (t')} + \hat{S}^\dagger \ev{\hat{R}(t')}
\end{equation}
The average of the reservoir operator over system time scales is effectively constant, by the first assumption, and the integral is trivial. 
If we further have a stationary reservoir fluctuation distribution, the first term vanishes altogether.
Otherwise, it contributes some amount to the normal hamiltonian evolution, with modified hamiltonian $\ham = \ham + \ham_{av}$

For the second term we expand the form of the interaction hamiltonian
\begin{align*}
  \mathscr{I}_2 &= -i\int_t^{t+\Delta t} \int_t^{t+\Delta t} \dd{t'} \dd{t''} \comm{\ham_I}{\comm{\ham_I}{\dens(t'')}}\\
                &= -i\int_t^{t+\Delta t} \int_t^{t+\Delta t} \dd{t'} \dd{t''} \\
                &\comm{\hres{t'}}{\comm{\hres{t''}}{\dens(t'')}}
\end{align*}
expanding out the commutator, and factorising the density operator, it is clear that the result will contain terms of the form $\tr{\hat{R}(t')\hat{R}(t'')\dens} = \ev{\hat{R}^\dagger(t') \hat{R}(t'')}$. Defining 
\begin{align}
  C^{--} &= \ev{\hat{R}(t') \hat{R}(t'')}\\
  C^{+-} &= \ev{\hat{R}^\dagger(t') \hat{R}(t'')}\\
  C^{-+} &= \ev{\hat{R}(t') \hat{R}^\dagger(t'')}\\
  C^{++} &= \ev{\hat{R}^\dagger(t') \hat{R}^\dagger(t'')}
\end{align}
The expanded integral
\begin{align}
  \mathscr{I}_2 =  &-i\int_t^{t+\Delta t} \int_t^{t+\Delta t} \dd{t'}\dd{t''} \Bigg\{\\
                   &C^{--}\Big(\hat{S}^\dagger \hat{S} \dagger \dens_S(t) - \hat{S}^\dagger \dens_S(t) \hat{S}^\dagger \Big)\\
                   &C^{+-}\Big(\hat{S}^\dagger \hat{S} \dens_S(t) - \hat{S}^\dagger \dens_S(t) \hat{S} \Big)\\
                   &C^{-+}\Big(\hat{S} \hat{S}^\dagger \dens_S(t) - \hat{S} \dens_S(t) \hat{S}^\dagger \Big)\\
                   &C^{++}\Big(\hat{S} \hat{S} \dens_S(t) - \hat{S} \dens_S(t) \hat{S} \Big) \Bigg\}
\end{align}
For the interaction with the electromagnetic vacuum, we have $\hat{R} = \ann_j,\ j\ \in\ \mathbb{N}$, for which $C^{++}=C^{--}=0$ and $C^{+-}=C^{-+}+1$.
To recover the Lindblad form, we perform the Wigner-Weisskopf approximation, as in i.e. \cite{Meystre2007} to perform the integral, and take the reservoir as a set of independent harmonic oscillators.
\begin{align}
  \dv{\dens_S}{t} &= \quad \ \ -\frac{\gamma}{2} \bar{n}(\omega) \left(\hat{S}\hat{S}^\dagger\dens_S - \hat{S}^\dagger \dens_S \hat{S}\right) + h.c. \\
                  &- \frac{\gamma}{2} (\bar{n}(\omega) + 1) \left(\hat{S}^\dagger\hat{S}\dens_S - \hat{S} \dens_S \hat{S}^\dagger\right) + h.c.
\end{align} 
Where $\bar{n}(\omega)$ s the thermal occupation of the reservoir at frequency $\omega$. 
It can be seen that this time evolution is trace-preserving \cite{Lukin2006}.
