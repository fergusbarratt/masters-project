\section{Circuit QED}
Circuit QED (cQED) is a promising paradigm for the implementation of quantum computation, as well as in quantum information and simulation.
The DiVincenzo Criteria \cite{DiVincenzo} provide a set of requirements for a scalable quantum computer. 
Of them, the most important in cQED is decoherence. 
\subsection{cQED qubits}
There are several scalable well-characterised qubit candidates in superconducting circuits \cite{Makhlin2001}.
They fall into distinct types based on the quantum degree of freedom they exploit: phase, charge, or flux.
Here we look at a particular type of charge qubit: the \emph{transmon}.
\begin{figure}[!htb]
  \centering
  \resizebox{.83\linewidth}{!}{\begin{tikzpicture}[scale=1]
    \draw (0, 0) -- (10, 0);

    \draw (0, 0) -- (0, 4);
    \draw (0, 6) -- (0, 10);

    \draw (0, 10) -- (4.5, 10);
    \draw (5.5, 10) -- (10, 10);

    \draw (10, 0) -- (10, 4);
    \draw (10, 6) -- (10, 10);

    \draw (9, 4) -- (9, 6);
    \draw (11, 4) -- (11, 6);
    \draw (9, 4) -- (11, 4);
    \draw (9, 6) -- (11, 6);
    \draw (9, 6) -- (11, 4);
    \draw (11, 6) -- (9, 4);

    \draw (4.5, 11) -- (4.5, 9);
    \draw (5.5, 11) -- (5.5, 9);

    \draw (0, 5) circle(1cm);

    \draw[red, thick, dashed] (5, 5) -- (5, 12) -- (12, 12) -- (12, 5) -- (5, 5);
\end{tikzpicture}
}
  \caption{Schematic of a Cooper Pair Box. Island is demarcated by the dashed line.}
  \label{cooperpairbox}
\end{figure}
\subsection{Cooper Pair Box}
The transmon is an improvement on the Cooper Pair Box(CPB) qubit.
In the CPB, a superconducting island is coupled through a Josephson junction and a capacitance to a voltage.
The system is described by the Hamiltonian \cite{Makhlin2001}
\begin{equation}
  \mathscr{H} = 4E_C(n-n_g)^2 - E_J \cos \Theta
\end{equation}
where n is the number of excess cooper pairs on the island, $\Theta$ the (conjugate) phase of the island superconducting order parameter, $E_J$ is the Josephson energy, $E_C$ the charging energy, and $n_g$ the dimensionless gate charge.
We can represent this hamiltonian in the charge basis:
\begin{align}
  \mathscr{H} &= \sum_n \Bigg\{ 4 E_C (n-n_g) \ketbra{n}{n}\\ 
              &- \frac{1}{2} E_J \left( \ketbra{n}{n+1} 
                                       + \ketbra{n+1}{n} \right) \Bigg\}
\end{align}
tuning $n_g$, we can select strongly just two states, and cast this hamiltonian into 2 level form.
\begin{equation}
  \mathscr{H} = 2E_C ( 1-2n_g ) \sigma_z - E_J \sigma_x 
\end{equation}
\subsection{Transmon}
There are two conflicting goals for an effective qubit. 
There must be a highly anharmonic level system, where one transition can be strongly addressed in isolation. 
Noise must be suppressed so that the qubit state can be faithfully read out and controlled. 
In the Cooper Pair Box, these goals are orthogonal.
\begin{figure}[!htb]
  \resizebox{1.0\linewidth}{!}{\documentclass{article}
\usepackage{tikz}

\begin{document}
  \begin{tikzpicture}[scale=0.8]
    \draw (0, 0) -- (11.5, 0);

    \draw (0, 0) -- (0, 4);
    \draw (0, 6) -- (0, 10);

    \draw (0, 10) -- (3.5, 10);
    \draw (4.5, 10) -- (11.5, 10);

    \draw (11.5, 0) -- (11.5, 3);
    \draw (11.5, 7) -- (11.5, 10);
    \draw (10, 3) -- (13, 3);
    \draw (10, 7) -- (13, 7);
    \draw (10, 7) -- (10, 6);
    \draw (13, 7) -- (13, 6);
    \draw (10, 3) -- (10, 4);
    \draw (13, 3) -- (13, 4);

    \draw (12, 4) -- (12, 6);
    \draw (14, 4) -- (14, 6);
    \draw (12, 4) -- (14, 4);
    \draw (12, 6) -- (14, 6);
    \draw (12, 6) -- (14, 4);
    \draw (14, 6) -- (12, 4);

    \draw (9, 4) -- (9, 6);
    \draw (11, 4) -- (11, 6);
    \draw (9, 4) -- (11, 4);
    \draw (9, 6) -- (11, 6);
    \draw (9, 6) -- (11, 4);
    \draw (11, 6) -- (9, 4);

    \draw (3.5, 11) -- (3.5, 9);
    \draw (4.5, 11) -- (4.5, 9);

    \draw (0, 5) circle(1cm);

    \draw[red, thick, dashed] (4, 5) -- (4, 12) -- (15, 12) -- (15, 5) -- (4, 5);

    \draw (5.5, 0) -- (5.5, 4.5);
    \draw (5.5, 10) -- (5.5, 5.5);

    \draw (4.5, 5.5) -- (6.5, 5.5);
    \draw (4.5, 4.5) -- (6.5, 4.5);
  \end{tikzpicture}
\end{document}
}
  \caption{Schematic of a transmon. New island is couped via capacitance C and another Josephson junction.}
  \label{anharmonicity}
\end{figure}
In \ref{anharmonicity} this behaviour is obvious. 
For low $\frac{E_J}{E_C}$, The greatest level anharmonicity is achieved by operating the qubit at the half integer charge degeneracy points.
At this point the gradient is zero and first order fluctuations are suppressed. 
However, the change in the transition width with small changes in gate charge is great. 
In contrast, for high $\frac{E_J}{E_C}$ this noise sensitivity disappears, at the cost of a great deal of level anharmonicity.
The transmon exploits the fact that the decrease of charge noise with increasing $\frac{E_J}{E_C}$ is exponential, while the decrease in anharmonicity is algebraic.
The transmon is a cooper pair box with an additional shunting capacitance $C_B$, and an additional parallel Josephson junction.
The shunting capacitance changes the noise behaviour of the device significantly.
While the transmon hamiltonian takes the same form as the CPB system, the additional capacitance suppresses the charging energy $E_C$ relative to the CPB.
This changed charging energy suppresses system noise at the cost of only a small amount of anharmonicity. \\

We take these qualities as those of a qubit, and couple the two level transmon system to a single electromagnetic field mode.
This is the Jaynes-Cummings system of Quantum Optics.
