\section{Methods}
\subsection{Analytical}
We consider the extent to which quantum fluctuations affect dynamics by deriving an effective mean-field theory.

Taking expectation values $\langle \dot{O} \rangle = \tr\{\dot{\rho} O \}$ of the operators $a, \atann, \sigma_z$ we obtain
\begin{align}
  \langle \dot{\ann} \rangle & = i(-\delta_{qd}\langle a \rangle - ig \langle \atann \rangle -i\Epsilon) -\kappa \langle a \rangle \\
  \langle \dot{\sigma_-} \rangle & = i(-\delta_{ad}\atann - ig \langle a \sigma_z \rangle) - \gamma \langle \atann \rangle \\
  \langle \dot{\sigma_z} \rangle & = -2g ( \langle \cre \sigma \rangle + \langle \ann \sigma^\dagger \rangle) - 2 \gamma \langle \sigma_z \rangle - 2 \gamma
\end{align}
we now make the assumption that all expectation values of products of operators factorize. This corresponds to the assumption that there are no correlations between different operators - this is patently non-physical, but the approximation yields equations in which the quantum fluctuation effects of these correlations are averaged out, and the qualitative behaviour remains the same to high order \cite{Jaynes1963a}. Defining $\langle \ann \rangle = \alpha, \langle \atann \rangle = \beta, \langle
\sigma_z \rangle = \zeta$, setting $\delta_{cd} = \delta_{qd}$ i.e. qubit-cavity resonance and adiabatically eliminating the drive, we attain what are known as the optical Bloch equations
\begin{align}
  \label{sc_with_gamma}
  \frac{d \alpha}{dt} &= -(\kappa - i \Delta \omega)\alpha - ig \beta - i\Epsilon\\
  \frac{d \beta}{dt} &= -(\frac{\gamma}{2}-i\Delta\omega)\beta+ig\alpha\zeta \\
  \frac{d\zeta}{dt} &= -\gamma (\zeta +1)+2ig(\alpha^*\beta-\alpha\beta^*) 
\end{align}
We consider the system without spontaneous emission in the text.
\begin{align}
  \label{sc_without_gamma}
  &\frac{d \alpha}{dt} = -(\kappa -i \Delta \omega) \alpha-ig \beta \\
  &\frac{d \beta}{dt} = i \Delta \omega \beta +ig \alpha \zeta \\
  &\frac{d \zeta}{dt} = 2 i g(\alpha^* \beta -\alpha \beta^*)
\end{align}
Since the length of the qubit pseudo-spin is conserved in the absence of qubit dissipation $\gamma$, we have also a fourth equation in this case.
\begin{equation}
  4|\beta|^2+\zeta^2 = 1 
\end{equation}
\subsubsection{Large n detuning approximation}
We also make use of the following approximation \cite{Alsing1990}
Given a driving field as above, the upper and lower path rungs (n, n+1, \dots) will be detuned from resonant drive by
\begin{align}
  \Delta E_u = \hbar g (\sqrt{n}-\sqrt{n-1}) \\
  \Delta E_l = -\hbar g (\sqrt{n}-\sqrt{n-1})
\end{align}
approximated for large n by
\begin{align}
  \Delta E_u &= \hbar g \sqrt{n} \left (1-\sqrt{\frac{n-1}{n}} \right ) \\
  &= \hbar g \sqrt{n} \left (1-\sqrt{1-\frac{1}{n}} \right ) \\
  & \approx \hbar g \sqrt{n} \left ( 1- \left ( 1 - \frac{1}{2n} \right ) \right ) \\
  &= \frac{\hbar g}{2 \sqrt{n}}
\end{align}
and
\begin{equation}
  \Delta E_l = -\frac{\hbar g}{2 \sqrt{n}}
\end{equation}
\subsection{Numerical}
Truncating the master equation density matrix Fock state expansion produces a finite set of coupled differential equations for the components, from which for low photon occupation expectation values in full quantum generality can be generated to high accuracy.
This method does however scale poorly in the system size (the number of elements in the density matrix $\propto N^2$, where N is the dimension of the system Hilbert space.
The method of quantum trajectories scales much better \cite{Molmer1993} $\propto N$ by considering the wavefunction stochastically).
Setting the time derivatives to zero here yields a matrix equation, which given the positive definiteness and hermiticity conditions on the density matrix yields to a Cholesky solver \cite{Press1992}.
We use the Matlab and Python solvers as exposed by QuTiP \cite{Johansson2013a} and qotoolbox \cite{Tan}. For the time dependent solutions, we also use fourth order runge-kutta methods provided in both libaries.
