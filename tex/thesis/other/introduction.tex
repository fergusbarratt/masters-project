\section{Introduction}
The Jaynes Cummings model is a fundamental model of quantum optics in which a single quantized electromagnetic field mode interacts with a two level system. 
The qubit introduces a fundamental nonlinearity into the simple harmonic dynamics of the single field mode. 

The Jaynes Cummings model has been the subject of many different theoretical investigations in many different guises, for example
\cite{Abdalla2011}
\cite{Benivegna1994}.
Coupled qubit cavity systems of this sort are interesting in application, and have been investigated in the context of open quantum systems, quantum measurement, and quantum information and computing.
One particular, promising implementation is that of circuit QED 
\cite{Blais2004a} 
\cite{Makhlin2000} 
\cite{Koch2007} 
where there are promising proposals for reaching the strong coupling regime of QED, for concrete investigations of the theory of open quantum systems, quantum measurement, and quantum information processing
\cite{You2003} 
\cite{Hood2000} 
\cite{Irish2003} .
This context also provides a base for broader investigation of effects in cavity QED 
\cite{Al-Saidi2002} 
\cite{Plastina2003} 
\cite{Marquardt2001} .

Here we investigate the driven-dissipative Jaynes Cummings Model using the typical parameter ranges of circuit QED, with a particular focus on bistabilities in the dispersive regime i.e. where the qubit transition frequency is far detuned from the cavity resonance.

The specific motivation is superconducting circuits as an implementation for quantum computation, specifically the \emph{transmon} qubit \cite{Koch2007}, a promising candidate for the suppresion of the CQED decoherence problem.
In the first section we review the theoretical background and recent results in this area, specifically the results on the JC system on resonance of \cite{Carmichael2015} and the dispersive results of \cite{Bishop2010}.

Finally, we present our results.
