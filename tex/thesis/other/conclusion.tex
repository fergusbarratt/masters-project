%% conclusion
\section{Conclusions}
We have considered the Jaynes-Cummings model in the resonant and strong dispersive regimes, using master equations, mean-field models, quasi-probability functions, and quantum trajectories. 
In the resonant case, following \cite{Carmichael2015}, we uncovered bimodality in phase and amplitude in different parts of the system parameter space. 
We analysed the steady state solutions to the mean field equations and investigated the degree to which the fully quantum behaviour matched with these solutions. 
We considered the effect of different loss mechanisms, and performed analysis of the system's approach to steady state. 
In the dispersive case, we compared results from the semiclassical model of \cite{Bishop2010}, with a model based on the mean-field equations. 
We performed an expansion in a quotient of the excitation number that holds for low excitation, and mapped the resulting system onto the quantum Duffing oscillator by freezing the qubit degree of freedom.
We analysed the degree to which this degenerate mapping describes the reality of the Jaynes-Cummings oscillator, by considering the effect of the participation of the qubit. 
We found that the participation of the qubit at small levels does not affect the dynamics of the Jaynes-Cummings oscillator so much as to break the correspondence between the JC system and the Duffing oscillator, given that the system remains in certain parameter regimes.
We analysed the extent of the region of bistability in parameter space, and compared this to the bistability exhibited by the Duffing oscillator, as described in \cite{Drummond1979}.

Following work would consider the generalised P function approach discussed in the background and consider its potential generalisation from the Duffing oscillator as derived in \cite{Drummond1979} to a closer approximation to the Jaynes-Cummings oscillator. 
Also interesting is the effect of the transmons extra levels on the dynamics of the oscillator, in particular whether the typical operating regimes of the device produce parameter regimes in which the richer level scheme affects the assumptions and correspondences we have made. 
